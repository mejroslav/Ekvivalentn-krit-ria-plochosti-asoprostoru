%--------------------------------------------------------------
%	REQUIRED PACKAGES
%--------------------------------------------------------------

\usepackage[
nochapters, % Turn off chapters since this is an article        
beramono, % Use the Bera Mono font for monospaced text (\texttt)
%eulermath,% Use the Euler font for mathematics
pdfspacing, % Makes use of pdftex’ letter spacing capabilities via the microtype package
dottedtoc % Dotted lines leading to the page numbers in the table of contents
]{classicthesis} % The layout is based on the Classic Thesis style

\usepackage{arsclassica} % Modifies the Classic Thesis package
\usepackage[T1]{fontenc} % Use 8-bit encoding that has 256 glyphs
\usepackage[utf8]{inputenc} % Required for including letters with accents
\usepackage[czech]{babel} % Český jazyk
\usepackage{graphicx} % Required for including images
%--------------------------------------------------------------
% Fonts
\usepackage{libertinus} % The Libertinus font

\usepackage{enumitem} % Required for manipulating the whitespace between and within lists
\usepackage{lipsum} % Used for inserting dummy 'Lorem ipsum' text into the template
\usepackage{subfig} % Required for creating figures with multiple parts (subfigures)



\usepackage{amsmath,amssymb,amsthm,amsfonts} % For including math equations, theorems, symbols, etc
\usepackage[czech]{varioref} % More descriptive referencing
\usepackage[top = 3.5 cm, bottom = 3.5 cm, left = 2.5 cm, right = 2.5 cm]{geometry}

\usepackage{mathtools}
\usepackage{amsmath}
\usepackage{float}
\usepackage{geometry}
\usepackage{csquotes}
\usepackage{caption}
\usepackage{color}
\usepackage{tensor}
\usepackage{mdframed}

%------------------------------------------------------------
%	DIAGRAMS AND TIKZ
\usepackage{smartdiagram}
\usepackage{metalogo}
\usepackage{tikz}
\usetikzlibrary{matrix,calc}

\usepackage{hhline} % kvůli double line v tabulkách


\usepackage{realhats}
%------------------------------------------------------------
%	THEOREM STYLES
%------------------------------------------------------------

\theoremstyle{definition} % Define theorem styles here based on the definition style (used for definitions and examples)
\newtheorem{definition}{Definice}
\newtheorem{example}{Příklad}
\newtheorem{exercise}{Cvičení}

\theoremstyle{plain} % Define theorem styles here based on the plain style (used for theorems, lemmas, propositions)
\newtheorem{theorem}{Věta}

\theoremstyle{remark} % Define theorem styles here based on the remark style (used for remarks and notes)
\newtheorem{remark}{Poznámka}


% TikZ
\usepackage{tikz}
\usepackage{pgfplots}
\usetikzlibrary{arrows.meta}
\usetikzlibrary{positioning}

%-------------------------------------------------------------
%	HYPERLINKS
%-------------------------------------------------------------

\hypersetup{
%draft, % Uncomment to remove all links (useful for printing in black and white)
colorlinks=true, breaklinks=true, bookmarks=true,bookmarksnumbered,
urlcolor=webbrown, linkcolor=RoyalBlue, citecolor=webgreen, % Link colors
pdftitle={}, % PDF title
pdfauthor={\textcopyright}, % PDF Author
pdfsubject={}, % PDF Subject
pdfkeywords={}, % PDF Keywords
pdfcreator={pdfLaTeX}, % PDF Creator
pdfproducer={LaTeX with hyperref and ClassicThesis} % PDF producer
}



%----------------------------------------------------
%	MATHEMATICS
%----------------------------------------------------

% Tělesa, obory íntegrity a metrické prostory
\newcommand{\C}{\mathbb{C}}
\newcommand{\R}{\mathbb{R}}
\newcommand{\N}{\mathbb{N}}
\newcommand{\Q}{\mathbb{Q}}
\newcommand{\Z}{\mathbb{Z}}
\renewcommand{\L}[2]{L^{#1} \left( #2 \right)} % Lebesgueovy prostory

\newcommand{\vc}[1]{\boldsymbol{#1}} % vektor
\newcommand{\mat}[1]{\mathbf{#1}} % matice

\newcommand{\norm}[1]{\left \Vert #1 \right \Vert} % norma vektoru
\newcommand{\set}[1]{ \left \lbrace #1 \right \rbrace} % množina
\newcommand{\const}{\mathrm{konst}} % konstanta

\newcommand{\F}{\mathcal{F} } % Fourierova transformace
\newcommand{\La}{\mathcal{L}} % Laplaceova transformace

% Označení funkcí
\newcommand{\Res}[2]{\mathrm{Res}_{#1} \, #2 \,} % residuum
\newcommand{\sgn}{\, \mathrm{sign} \,} % signum
\newcommand{\tg}{\,\mathrm{tg}\,} % možné značení tangens


%Značení derivací a integrálů
\newcommand{\der}[2]{\frac{\mathrm{d}#1}{\mathrm{d}#2}} % obyčejná derivace
\newcommand{\pder}[2]{\frac{\partial #1}{\partial #2}} % parciální derivace
\newcommand{\ader}[2]{\frac{\mathrm{D} #1}{\mathrm{d} #2}} % absolutní derivace
\newcommand{\apder}[2]{\frac{\mathrm{D} #1}{\partial #2}} % absolutní parciální derivace

\newcommand{\tder}[3]{\left( \pder{#1}{#2} \right)_{#3 = \const}} % termodynamická derivace
\newcommand{\D}{\mathrm{d} } % integrační znamení
\newcommand{\DD}{\mathrm{D}} % absolutní derivace
\newcommand{\intR}{\int_{-\infty}^{\infty}} % integrál přes reálnou osu



% Značení posloupností, limit a sum
\newcommand{\sequence}[2]{ \left \lbrace #1 \right \rbrace_{#2=1}^\infty} % posloupnost
\newcommand{\sumnorm}[1]{\sum_{#1}^\infty} 
\newcommand{\limplus}[1]{\lim_{#1 \rightarrow + \infty}}
\newcommand{\limminus}[1]{\lim_{#1 \rightarrow - \infty}}
\newcommand{\limn}{\lim_{n \rightarrow \infty}}
\newcommand{\LH}{\overset{l'H}=}


% VŠE záležitosti
\newcommand{\dder}[2]{\frac{\Delta #1}{\Delta #2}}


\newcommand{\arctg}{\mathrm{arctg}\,}
\newcommand{\cotg}{\mathrm{cotg}\,}
\newcommand{\arccotg}{\mathrm{arccotg}\,}

